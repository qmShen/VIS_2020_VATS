\section{Related work}
The most related techniques to our work include the visual analysis of trajectory dataset, the methodology of large data visualization and data sampling. 
\subsection{Trajectory analysis}
Trajectory, consisting of a sequence of spatial locations, is the most common form of the object movement. To support the understanding and analysis of the trajectory dataset, many visualization and visual analytics system are developed. The detailed summary of these work is presented in~\cite{chen2015survey}. These techniques can be classified into three categories according to visualization form: point-based visualization, line-based visualization and region-based visualization. 

The point-based visualization capture the basic spatial distribution of the passing points of the moving object\cite{}. Furthermore, many density-based methods such as the kernel density estimation(KDE) are applied based on the point-based visualization, by the sacrifice of the detail the information of trajectories, these methods alleviate the visual clutter caused by large amount of data. 
Furthermore, to be better applied in the city environment, advanced KDE techniques are developed to capture the moving patterns along the road networks~\cite{}. In the study of urban traffic, the point-based visualization can capture the hot regions, but unable to identify the movement of the individual case and reveal the moving information such as the direction and origin-destination ~\cite{chen2015survey}. Line-based techniques are the most commonly used visualization methods which present the trace of the movement as polylines, thus to preserve the continuous moving information~\cite{}. However, due the large amount of the trajectories, the line-based methods always cause serious visual clutter due to the cross of the polylines~\cite{}. To alleviate this problem, the clustering techniques are applied in the preprocessing step before visualization and then visualize the clusters by ribbon ~\cite{},  glyph~\cite{} or sankey diagrams~\cite{}. Moreover, advanced interaction techniques~\cite{}, sampling techniques~\cite{} and edge bundling techniques~\cite{zeng2019route} are also developed to better present the movement patterns. The region based techniques divide the whole region into sub-regions in advance and then visualize the traffic situation before the sub-regions. These methods visualize the macro-pattern very well by leveraging different aggregation techniques such as the administrative regions~\cite{}, uniform grid~\cite{} and spatial clustering results~\cite{}.


\subsection{Large data visualization}
Another topic related to our work is the large data visualization. The movement dataset, such as the urban traffic, always contains millions to billions of trajectories, which is challenge to analyze and visualize. Three methods are applied to support the visual analytics of large datasaet: interactive method, aggregation method and reduction method. All these methods aim to reduce the amount of the rendering items to support the efficient visualization. 

\QM{More discussion about the limitations}

\subsection{Data sampling techniques}
As we have discussed in the previous sections, the sampling methods reduce the data volume by directly selecting the representative items. Current advancing sampling techniques in the visualization domain are mostly designed for the scatter plot and aim to not only solve the overdrawing of the points but also try to preserve the information distribution of the data items. 
Some works design advanced sampling algorithms to preserve the meaningful data items according to the analyzing requirement such as the mulit-class data analysis~\cite{chen2014visual} and hierarchical exploration~\cite{}. Furthermore, to the usage of more visual channels of the points other than location such as color~\cite{chen2014visual}, size~\cite{woodruff1998constant} and opacity are discussed. 
Closely related to our work, Park et al.~\cite{park2016visualization} proposed the visualization-aware techniques for the scatter plot. They proposed visualization-inspired loss which effectively evaluates the visual loss of the sampling result and validates the proposed method based on three common visualization goals:  regression, density estimation and clustering. 

In comparison with the sampling techniques for scatter plot, the trajectory sampling is more challenging because of the complexity of the trajectories~\cite{pelekis2010unsupervised}. \QM{Most} of the existing trajectory sampling techniques cluster the trajectories first~\cite{pelekis2009clustering} and then select the most representative trajectories from each cluster, which highly depend on the distance calculation~\cite{pelekis2007similarity} and clustering algorithms. Some techniques further focus on the clustering and sampling of trajectory segments instead of the whole trajectories~\cite{panagiotakis2011segmentation}. 
\QM{None of these sampling techniques are proposed from the motivation visualization-aware...}