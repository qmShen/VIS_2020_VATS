\section{Discussion}

The experiment results and the user studies demonstrate that $\avats$ outperforms the uniform random sampling in guaranteeing the visual quality and enriching the flow information. According to the user feedback and our owe experience, the parameters $\delta$ will affect the visualization result across different map resolutions. Furthermore, we also notice that the proposed method also performs differently by various dataset. In this section, we will discuss these issues to clarify the characteristics of the algorithm and explain how these parameters make our method more effective.

\stitle{Representative parameter} 
Generally speaking, the data sampling inevitably results in the information loss because only partial data are leveraged to generate the visualization. A well designed sampling algorithm for specific exploration tasks aims to preserve as much important information as possible with a limited size of sub-dataset. We have discussed the limitation of uniform random sampling technique in section(***). A clear drawback of random sampling is that the region with low data distribution tend to have fewer preserved data, which may significantly influence of perception of human beings and reduce the visual quality. Our proposed solution $\avats$ reduce the selection of the dense region by proposing idea of trajectory rejection that is a chosen trajectory may reject other nearby trajectories. An intuitive understanding for this parameters is that it indicates how close a candidate trajectories should be rejected. If the parameters is large, more trajectories in the dense region will be rejected and more trajectories in the sparse region will be preserved. While if the parameters is small, more trajectories at the dense region will be preserved. According to our user study, user explore the trajectory data mostly at the \QM{dense regions}, thus the recommended parameter setting is around \QM{64}. The users can also adjust the parameters according their owe requirement, a general principle is that the larger $delta$ and smaller $delta$ will be helpful for the exploration in sparse region and dense regions, respectively. Moreover, if the users focus on the discovering the general patterns from the overview, a larger parameters is recommended.

\stitle{Approach usability}

The sampling method only designed for visualization purpose