\section{Problem Statement}\label{sec:pro}
In this section, we first define our research problem in Section~\ref{sec:def} formally.
Section~\ref{sec:hard} analyzes the hardness of our research problem.
%Visualizing a large collection of trajectories are used frequently in map service or smart city applications.
%When exploring a large collection of trajectories, efficient and effective large-scale trajectory visualization is challenging in both academia and industry.
%The reasons are (i) the size of trajectory data is very large (e.g., several GB in an hour),
%and (ii) the limited rendering ability of existing commercial graphics device (e.g., XXX).
%
%A naive solution to employ sampling idea for large-scale trajectory visualization problem is randomly selecting several trajectories from the data set then visualize it by graphics device.
%However, the visualization result may be not acceptable by the user.
%In this work, we study the
%In particular, we focus on visual quality guaranteed sampling method for large trajectory data visualization.
%The major challenges to design visual quality guaranteed sampling method are:
%(I) how to define visual quality theoretically? (II) how to guarantee the quality of the sampling-based visualization result?
%In this work, we first formate visual quality by defining the loss function between the visualization results of the whole dataset and sampled dataset.
%With the loss function, we analyze the hardness of the problem, and devise a visual quality guaranteed sampling algorithm for it.


\subsection{Problem Definition}\label{sec:def}
As we analyzed in Section~\ref{sec:intro}, the large-scale (e.g., hundreds of millions GPS points) line-based trajectory visualization problem is very challenging due to the large data size and limited rendering capability of graphics devices.
To make matters worse, the visualization result of large-scale trajectory dataset suffers visual clutter seriously.
In this work, we focus on how to visualize large-scale trajectory dataset efficiently and effectively.
In particular, our objective is to devise a visual fidelity guaranteed sampling method for large trajectory data visualization.
The major challenges to achieve this goal are:
(i) how to define visual fidelity theoretically? (ii) how to guarantee the visual fidelity of the sampling-based visualization result?
We commence our presentation by defining our research problem formally as follows.

\begin{problem}[Large-scale trajectory visualization problem]\label{prob:def}
Given a large-scale trajectory dataset $\D$ and a sampling rate $\alpha$,
the trajectory visualization problem is selecting a subset of trajectories $\oR \subseteq \D$ with $|\oR| \leq \alpha |\D|$,
such that visual fidelity loss function $loss(\oR,\D)$ is minimized.
\end{problem}

%From the user’s perspective, there are many ways to define the loss function $loss$ between the visualization result qualities of the sampled subset $\oR$ and the whole dataset $\D$.
%For example, \cite{park2016visualization} defined point-based loss function for very large scatter points visualization.
%However, it is not applicable for trajectory data visualization.
%In order to address that, we propose an novel loss function for trajectory visualization problem.


The key to solving the large-scale trajectory visualization problem (see Problem~\ref{prob:def}) is defining the visual fidelity loss function properly.
Intuitively, the visual fidelity of the sampled visualization results $\oR$ w.r.t. the original dataset $\D$ depends on the user specified visualization level of details (a.k.a., LOD).
Given an empty canvas (e.g., displaying device) with a user specified level of details,
the visualization process is rendering the trajectories into canvas with the given level of details, e.g., the number of pixels in each row and each column.
Considering a trajectory data set $\D$ and a subset of trajectories $\oR \subseteq \D$,
the visual fidelity loss between $\oR$ and $\D$ is defining as the different pixels of the visualization results about $\oR$ and $\D$ in the canvas with specified LOD.
We then define the visual fidelity loss function of sampling-based trajectory visualization problem as $loss(\D, \oR) = \frac{|\V(\D) - \V(\oR)|}{|\V(\D)|}$,
where $\V()$ measures the rendered pixels in the canvas of the input trajectory dataset.

Therefore, given a trajectory data set $\D$ and a sampling rate $\alpha$,
our research objective is finding subset $\oR$, such that  the visualization fidelity loss function $loss(\D, \oR)$ is minimized:
$$ \min_{\oR \subseteq \D, |\oR| \leq \alpha |\D|}  loss(\D, \oR) =  \frac{|\V(\D) - \V(\oR)|}{|\V(\D)|}. $$ %\\ %\nonumber


%  \Leftrightarrow  & \min_{\oR \subseteq \D, |\oR| = k}  \V(\D) - \V(\oR) \\ \nonumber
%  \Leftrightarrow  & \min_{\oR \subseteq \D, |\oR| = k}  \sum_{\forall \D_i \in \D} \V(\D_i) - \V(\oR)
%Intuitively, the visualization quality from user's perspective highly depends on the resolution of displaying device,
%e.g., the number of distinct pixels in each dimension that can be displayed.

%Given a high resolution displaying device, data objects set $\D$, and a subset of data objects $\oR \subseteq \D$.
%The visualization quality loss between $\oR$ and $\D$ is defining as the different pixels of the visualization results of $\oR$ and $\D$ in the given display device.
%Given a trajectory data set $\D$ and an integer $k$,  our research objective is selecting a sized-$k$ subset of trajectories which minimize
%the visualization quality loss function $loss(\D, \oR)$.
%Formally, the loss function is defined as $loss(\D, \oR) = \V(\D) - \V(\oR)$.



\subsection{Hardness Analysis}\label{sec:hard}
For the sake of presentation, we analyze the hardness of our research objective with a simple render manner of visualization result.
We are aware there exist complex rendering schemes, e.g., different pixels has different colors, we will consider them shortly.
In particular, for each pixel in the canvas with simple render manner, it will be rendered if there is a trajectory pass through it, otherwise it will not be rendered.
Suppose each pixel in the canvas has an unique id, let $\U$ be the universal set of all pixels in the canvas.
For each trajectory $t_i \in \D$, it consists of a set of pixels in the canvas.
In other words, the pixel set of each trajectory $t_i \in \D$ is a subset of $\U$.
Consequently, the pixel set of the selected trajectory set $\oR$ is also a subset of $\U$ as $\oR = \cup_{t_i \in \oR} t_i$.


Our research objective is minimizing loss function $loss(\D, \oR) =  \frac{|\V(\D) - \V(\oR)|}{|\V(\D)|}$ subject to $\oR \subseteq \D$ and $|\oR| \leq \alpha |\D|$.
Given an empty canvas, the visualized/rendered pixels of the input trajectory dataset $\D$ is a constant set, denotes as $\mathsf{C}$.

Hence, our research objective of Problem~\ref{prob:def} can be transformed as follows:
\begin{align}\label{eqn:obj2} \nonumber
\text{Objective}: & \min_{\oR \subseteq \D, |\oR| = \alpha |\D|}  \frac{|\V(\D) - \V(\oR)|}{|\V(\D)|} \Leftrightarrow \min_{\oR \subseteq \D, |\oR| = \alpha |\D|}  \frac{|\mathsf{C} - \V(\oR)|}{|\mathsf{C}|}  \\ \nonumber %
& \Leftrightarrow \min_{\oR \subseteq \D, |\oR| = \alpha |\D|}   - |\V(\oR)| \Leftrightarrow \max_{\oR \subseteq \D, |\oR| = \alpha |\D|}  |\V(\oR)|  \\ \nonumber
& \Leftrightarrow \max_{\oR \subseteq \D, |\oR| = \alpha |\D|} | \cup_{t_i \in \oR} t_i |
\end{align}

It is equivalent to the well-known set cover maximization problem.
Specifically, given an integer $k = \alpha |\D|$, and a collection trajectory pixel set $\D = \{t_1, t_2, \cdots, t_n \}$ with $\forall t_i \subset \U$,
the objective of the set cover maximization problem is finding a subset $\oR \subset \D$ such that $|\oR| \leq k$ and the number of covered pixels in $|\cup_{t_i \in \oR} t_i|$ is maximized.
We omit the proof of its NP-hardness, as it has been shown in~\cite{setcover}.


%It is equivalent to select sized-$k$ trajectory set $\oR$ from $\D$ which $\cup_{\oR_i \in \oR} \oR_i$ is maximized.
%It is a NP-hard problem as we proved in Lemma~\ref{lem:np}.

%\begin{lemma}[NP hard]~\label{lem:np}
%Given a trajectory dataset $\D$ and an integer $k$,
%The sampling-based trajectory visualization problem (see Problem~\ref{prob:def}) is NP-hard.
%\end{lemma}

%We omit the proof of Lemma~\ref{lem:np} as it is a typical set cover maximization problem\footnote{\url{https://en.wikipedia.org/wiki/Maximum_coverage_problem}}, which is a well-known NP-hard problem in literature.

